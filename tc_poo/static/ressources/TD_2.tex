\documentclass[12pt]{article}
\usepackage[utf8]{inputenc}

\usepackage[pdftex,a4paper,bookmarks]{hyperref}
\usepackage{fullpage}


\usepackage{url}
\usepackage{listings}
\usepackage{courier}
\lstset{basicstyle=\ttfamily}


\begin{document}


    \begin{center}
      \begin{tabular}{c}
      \hline
    \\
        {\bf \textsf {\Large Programmation Orientée Objet}}\\
    \\
        {\bf \textsf {\Large TD 2 : Composition et agrégation}}\\
    \\
        \hline
      \end{tabular}
    \end{center}
    \vspace{15mm}

\section{Fonctions et namespaces}

\subsection{Les fonctions sont des variables comme les autres}


\lstset{language=Python}
\begin{lstlisting}
une_liste = []
truc = une_liste.append
truc("?")

print(une_liste)
\end{lstlisting}


Dans le code ci-dessus, que représente \verb|append|~? Exécutez-le en montrant toutes les lignes de codes et les {\em namespaces} utilisés.


\subsection{Fonctions de fonctions}

On souhaite créer une fonction \verb|ajoute| avec un paramètre (entier) \verb|x|. Le retour de cette fonction doit être une fonction à un paramètre \verb|y| qui rend \verb|x + y|

\subsubsection{Tests}

Comment testeriez vous cette fonction~?

\subsubsection{Implémentation}

Codez cette fonction et vérifiez (à la main) qu'elle passe bien vos tests en notant toutes les variables et les namespaces rencontrés.

\section{Des Dés}

Nous allons réutiliser les dés de la séance 1. Pour pouvoir jouer à des jeux de dés, implémentons une classe
\verb|TapisVert|. Cette classe doit avoir~:
\begin{itemize}
	\item 5 dés comme attribut,
	\item pouvoir lancer les 5 dés simultanément ou individuellement,
	\item Connaître la valeur d'un dé spécifique.
\end{itemize}

Quel lien la classe \verb|TapisVert| a-t-elle avec la classe \verb|Dice| ?\\
Proposez un modèle UML pour cette classe.\\
Écrivez quelques tests unitaires permettant de vérifier son bon fonctionnement.\\

\section{Des Cartes}
\subsection{Une carte}
Donnez le diagramme UML d'une classe \verb|Card| définie par une couleur et une valeur. Les couleurs possibles sont communes à
toutes les cartes.

\subsection{Un tas de cartes}
Un \verb|Deck| est un tas de cartes, initialement vide auquel on peut ajouter une carte, dont on peut voir la carte du
dessus et dont on peut prendre la carte du dessus (la piocher i.e la récupérer et l'enlever du paquet). Proposez un
diagramme UML pour cette classe. Quel est son lien avec la classe \verb|Card| ?

\end{document}

%%% Local Variables:
%%% mode: latex
%%% TeX-master: t
%%% End:
